\documentclass[16pts]{report}
\usepackage[utf8]{inputenc}
\usepackage[T1]{fontenc}
\usepackage[francais]{babel}
\usepackage{xcolor}
\usepackage{url}
\usepackage{amsmath}
\usepackage{graphicx}
\usepackage{geometry}
\geometry{hmargin=2.5cm,vmargin=1.5cm}
\usepackage{etoolbox}


\usepackage{listings}

\lstdefinestyle{customc}{
  belowcaptionskip=1\baselineskip,
  breaklines=true,
  frame=L,
  xleftmargin=\parindent,
  language=C,
  showstringspaces=false,
  basicstyle=\footnotesize\ttfamily,
  keywordstyle=\bfseries\color{green!40!black},
  commentstyle=\itshape\color{purple!40!black},
  identifierstyle=\color{blue},
  stringstyle=\color{orange},
}

\lstset{escapechar=@,style=customc}

\lstset{language=C++,texcl=true}
\makeatletter
% First, modify the \@endpart macro.
\def\@endpart{}

% Next, copy the \chapter macro to \nonewpagechapter, and ...
% ... suppress page-breaking instructions in the modified macro
\let\nonewpagechapter\chapter
\patchcmd\nonewpagechapter{\if@openright\cleardoublepage\else\clearpage\fi}{}{}{}

\title{Rapport du Projet\\ Compilation}
\author{Aupetit Jordan, Thiaolayel Bruno}
\date{10 décembre 2013}


\begin{document}

\maketitle
\clearpage

\chapter{Spécifications}

\section{Conventions}

\begin{lstlisting}
/********  Types simples  ********/
int a;              //Entier signé codé sur 2 octets
short a;            //Entier signé codé sur 1 octet

unsigned int a;     //Entier non-signé codé sur 2 octets
unsigned short a;   //Entier non-signé codé sur 1 octet

bool a;             //Booléen codé sur 1 octet

char a = 'a';             //Caractère codé sur 4 octets
float a;            //Réel sur 4 octets

enum couleur {rouge, vert, bleu} //Enumération sur 1 octet


/******** Types complexes ********/
int t[4..3]; //Intervalles matchés mais non générés. (cf plus loin)
string a = "bonjour"; //Chaîne de 'n' caractère

type tab[n]; //Avec 'type' quelconque, et 'n' sa taille en entier

type* a; //Avec 'type' quelconque
\end{lstlisting}


\begin{lstlisting}
/******** Exemple de fichier input ********/

//Déclaration et affectations
int a = 0;
int b, c, d, e;
b = 2;
c = 3 - 2%2 + 4/23;
d = 4;
e = 5;
f = true;

//Tests et boucles
if (b < (c || f)) {
    for (int w = 30; w > 10; w = w - 10) {
        j = j + 1;
    }
} else if (c && d)
    while(true)
        e = e + 1;

bool test = true;
//Blocs
{
    int tutu;
    int k = 4;
    while(k)
        for (int j= 0; test; j = j + 3);
            test = !test;

    {
        int tutu;
        int k = 4;
        while(k)
            for (int j= 0; !test; j = j + 3);
                test = !test;
    }
}

/*****/

// Procédure

//Ne renvoyant rien :
toto(int a, float b, char c, int* d) {
    //TODO
    do{
        j = k; //TODO
    }while (true);

}

//Renvoyant une valeur :
titi(int a, float b, char c, int* d) {
    //TODO

} return b;

toto(a, b, c, d); //TODO

\end{lstlisting}




\section{Grammaire sous forme de Backus-Naur}

\begin{lstlisting}
start with beginning;

beginning ::= beginning instruction
            | /* Fin du parsing */
            ;

instruction ::= declaration
             |  affectation
             |  enumeration
             |  instruction_for
             |  instruction_while
             |  instruction_if
             |  bloc_instruction
             |  procedure
             ;

declaration ::= ENUM_TYPE liste_expression
             ;

affectation ::= ENUM_TYPE expression_variable:id AFFECT expression:exp SEMIC
             |  expression_variable:id AFFECT expression:exp SEMIC

             |  POINTER ENUM_TYPE expression_variable:id AFFECT expression:exp SEMIC
             ;

enumeration ::= ENUM_TYPE expression_variable:id LBRACE liste_expression:list RBRACE SEMIC
             ;

instruction_if ::= IF expression:condition THEN
                      instruction:blocIf
                      ELSE instruction:blocElse

                |  IF expression:condition THEN
                      instruction:blocIf
                ;

instruction_for ::= FOR LRBRA affectation:affect expression:condition SEMIC increment:inc RRBRA
                        instruction:blocFor
                 ;


increment ::= expression_variable:id AFFECT expression:exp
           |
           ;

instruction_while ::= WHILE expression:condition
                         instruction:blocWhile
                   |  DO
                        instruction:blocWhile
                      WHILE expression:condition SEMIC
                   ;


procedure ::= ID:nom LRBRA liste_expression:parametres RRBRA
                      bloc_instruction:blocInstruction
                      valeur_retour:ret
            | ID:nom LRBRA liste_expression:parametres RRBRA
                      bloc_instruction:blocInstruction
            | ID:nom LRBRA liste_expression:parametres RRBRA SEMIC
            ;

valeur_retour ::= RETURN expression SEMIC
               ;

bloc_instruction ::= LBRACE liste_instruction RBRACE
                  ;

liste_instruction ::= liste_instruction instruction
                   |
                   ;

liste_expression ::= liste_expression COMMA expression
                  |  expression
                  ;

expression ::= expression PLUS   expression
            |  expression MINUS  expression
            |  expression TIMES  expression
            |  expression DIVIDE expression
            |  expression MOD    expression
            |  MINUS expression

            |  expression AND     expression
            |  expression OR      expression
            |  expression DIFF    expression
            |  expression LT      expression
            |  expression GT      expression
            |  expression GE      expression
            |  expression LE      expression
            |  expression EQ      expression

            |  NOT expression
            |  LRBRA expression RRBRA

            |  expression_variable
            |  expression_valeur
            ;

expression_valeur ::= INT
                   |  REAL
                   |  BOOL
                   |  CHARACTERS
                   |  CHARACTER
                   ;

expression_variable ::= ID
                     |  ID LSBRA expression RSBRA
                     |  ID LSBRA expression INTER expression RSBRA
                     |  TIMES ID //Pointeur
                     ;

\end{lstlisting}
A noter : JUMPBOOL : optimisation des sauts conditionnels : arret des tests au premier faux.







\end{document}
